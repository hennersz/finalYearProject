%%%%%%%%%%%%%%%%%%%%%%%%%%%%%%%%%%%%%%%%%
% Arsclassica Article
% LaTeX Template
% Version 1.1 (10/6/14)
%
% This template has been downloaded from:
% http://www.LaTeXTemplates.com
%
% Original author:
% Lorenzo Pantieri (http://www.lorenzopantieri.net) with extensive modifications by:
% Vel (vel@latextemplates.com)
%
% License:
% CC BY-NC-SA 3.0 (http://creativecommons.org/licenses/by-nc-sa/3.0/)
%
%%%%%%%%%%%%%%%%%%%%%%%%%%%%%%%%%%%%%%%%%

%----------------------------------------------------------------------------------------
%  PACKAGES AND OTHER DOCUMENT CONFIGURATIONS
%----------------------------------------------------------------------------------------

\documentclass[
10pt, % Main document font size
a4paper, % Paper type, use 'letterpaper' for US Letter paper
oneside, % One page layout (no page indentation)
%twoside, % Two page layout (page indentation for binding and different headers)
headinclude,footinclude, % Extra spacing for the header and footer
BCOR5mm, % Binding correction
]{scrartcl}

\input{structure.tex} % Include the structure.tex file which specified the document structure and layout

\hyphenation{Fortran hy-phen-ation} % Specify custom hyphenation points in words with dashes where you would like hyphenation to occur, or alternatively, don't put any dashes in a word to stop hyphenation altogether

%----------------------------------------------------------------------------------------
%  TITLE AND AUTHOR(S)
%----------------------------------------------------------------------------------------

\title{\normalfont\spacedallcaps{A Decentralised Public Key Infrastructure and Messaging System.}} % The article title

\author{\spacedlowsmallcaps{Henry Mortimer}} % The article author(s) - author affiliations need to be specified in the AUTHOR AFFILIATIONS block

\date{} % An optional date to appear under the author(s)

%----------------------------------------------------------------------------------------

\begin{document}

%----------------------------------------------------------------------------------------
%  HEADERS
%----------------------------------------------------------------------------------------

\renewcommand{\sectionmark}[1]{\markright{\spacedlowsmallcaps{#1}}} % The header for all pages (oneside) or for even pages (twoside)
%\renewcommand{\subsectionmark}[1]{\markright{\thesubsection~#1}} % Uncomment when using the twoside option - this modifies the header on odd pages
\lehead{\mbox{\llap{\small\thepage\kern1em\color{halfgray} \vline}\color{halfgray}\hspace{0.5em}\rightmark\hfil}} % The header style

\pagestyle{scrheadings} % Enable the headers specified in this block

\maketitle % Print the title/author/date block

%----------------------------------------------------------------------------------------
% Aims and Objectives
%----------------------------------------------------------------------------------------

\section{Aims and Objectives} % This section will not appear in the table of contents due to the star (\section*)

\subsection{Aims}
To learn more about networking, more specifically peer to peer networks and improve my knowledge of public key cryptography, key exchange and key signing protocols. Use this knwoledge to produce a better way to distribute public keys.

\subsection{Objectives}

\begin{enumerate}
  \item Review information on peer to peer protocols such as "chord" and current strategies for exchanging and signing public keys
  \item Build a peer to peer network.
    \begin{enumerate}
      \item It must allow peers to quickly discover each other.
      \item It must allow peers to exchange messages.
    \end{enumerate}
  \item Implement a key signing and exchange system based on the GPG web of trust.
    \begin{enumerate}
      \item It must facilitate peers signing keys for other peers.
      \item It must allow peers to lookup keys using some identifier (e.g email).
      \item Peers should be able to make some effort to mark key as invlaid.
    \end{enumerate}
  \item Create a simple messaging system that uses the network to encrypt and sign messages.
    \begin{enumerate}
      \item Should work seamlessly i.e messages are automatically encyrpted, sign decrypted etc without interaction from the user.
    \end{enumerate}
  \item Evaluate the network based on properties such as speed of key lookup and authenticity of results. 
\end{enumerate}

\section{Deliverables}
  \begin{enumerate}
    \item A protocol specification for the key exchange and signing system
    \item A simple messaging system or plugin for an existing message system that implements the protocol as a proof of concept
    \item A report detailing testing of the system and reviewing the results and performance of the system. 
  \end{enumerate}
\section{Work Plan}
  
\begin{itemize}
  \item Project start to end October (4 weeks). Research peer to peer networks and public key infrastructure. Produce a list of resources for reference during the next stages.
  \item Mid-October to mid-November (4 weeks). Refine your requirements and start the initial iteration(s).
  \item November (16 weeks) to mid February. Work through Iterations.
    \begin{enumerate}
      \item Produce initial simple peer to peer system prototype to test setting up connections between various vertual machines.(1 week)
      \item Implement peer to peer protocol in full.(5 weeks)
      \item Integrate key insertion and lookup methods into protocol.(4 weeks)
      \item Implement messaging application or plugin.(4 weeks)
      \item Test and review functionality and performance of the protocol and application.(2 weeks)
    \end{enumerate}
  \item Mid-February to end of March (6 weeks). Work on Final Report
\end{itemize}
  

% %----------------------------------------------------------------------------------------
% %  AUTHOR AFFILIATIONS
% %----------------------------------------------------------------------------------------

% {\let\thefootnote\relax\footnotetext{* \textit{Department of Biology, University of Examples, London, United Kingdom}}}

% {\let\thefootnote\relax\footnotetext{\textsuperscript{1} \textit{Department of Chemistry, University of Examples, London, United Kingdom}}}

% %----------------------------------------------------------------------------------------

% \newpage % Start the article content on the second page, remove this if you have a longer abstract that goes onto the second page

% %----------------------------------------------------------------------------------------
% %  INTRODUCTION
% %----------------------------------------------------------------------------------------

% \section{Introduction}

% A statement\footnote{Example of a footnote} requiring citation \cite{Figueredo:2009dg}.

% \lipsum[1-3] % Dummy text

% Some mathematics in the text: $\cos\pi=-1$ and $\alpha$.
 
% %----------------------------------------------------------------------------------------
% %  METHODS
% %----------------------------------------------------------------------------------------

% \section{Methods}

% Reference to Figure~\vref{fig:gallery}. % The \vref command specifies the location of the reference

% \begin{figure}[tb]
% \centering 
% \includegraphics[width=0.5\columnwidth]{GalleriaStampe} 
% \caption[An example of a floating figure]{An example of a floating figure (a reproduction from the \emph{Gallery of prints}, M.~Escher,\index{Escher, M.~C.} from \url{http://www.mcescher.com/}).} % The text in the square bracket is the caption for the list of figures while the text in the curly brackets is the figure caption
% \label{fig:gallery} 
% \end{figure}

% \lipsum[5] % Dummy text

% \begin{enumerate}[noitemsep] % [noitemsep] removes whitespace between the items for a compact look
% \item First item in a list
% \item Second item in a list
% \item Third item in a list
% \end{enumerate}

% %------------------------------------------------

% \subsection{Paragraphs}

% \lipsum[6] % Dummy text

% \paragraph{Paragraph Description} \lipsum[7] % Dummy text

% \paragraph{Different Paragraph Description} \lipsum[8] % Dummy text

% %------------------------------------------------

% \subsection{Math}

% \lipsum[4] % Dummy text

% \begin{equation}
% \cos^3 \theta =\frac{1}{4}\cos\theta+\frac{3}{4}\cos 3\theta
% \label{eq:refname2}
% \end{equation}

% \lipsum[5] % Dummy text

% \begin{definition}[Gauss] 
% To a mathematician it is obvious that
% $\int_{-\infty}^{+\infty}
% e^{-x^2}\,dx=\sqrt{\pi}$. 
% \end{definition} 

% \begin{theorem}[Pythagoras]
% The square of the hypotenuse (the side opposite the right angle) is equal to the sum of the squares of the other two sides.
% \end{theorem}

% \begin{proof} 
% We have that $\log(1)^2 = 2\log(1)$.
% But we also have that $\log(-1)^2=\log(1)=0$.
% Then $2\log(-1)=0$, from which the proof.
% \end{proof}

% %----------------------------------------------------------------------------------------
% %  RESULTS AND DISCUSSION
% %----------------------------------------------------------------------------------------

% \section{Results and Discussion}

% \lipsum[10] % Dummy text

% %------------------------------------------------

% \subsection{Subsection}

% \lipsum[11] % Dummy text

% \subsubsection{Subsubsection}

% \lipsum[12] % Dummy text

% \begin{description}
% \item[Word] Definition
% \item[Concept] Explanation
% \item[Idea] Text
% \end{description}

% \lipsum[12] % Dummy text

% \begin{itemize}[noitemsep] % [noitemsep] removes whitespace between the items for a compact look
% \item First item in a list
% \item Second item in a list
% \item Third item in a list
% \end{itemize}

% \subsubsection{Table}

% \lipsum[13] % Dummy text

% \begin{table}[hbt]
% \caption{Table of Grades}
% \centering
% \begin{tabular}{llr}
% \toprule
% \multicolumn{2}{c}{Name} \\
% \cmidrule(r){1-2}
% First name & Last Name & Grade \\
% \midrule
% John & Doe & $7.5$ \\
% Richard & Miles & $2$ \\
% \bottomrule
% \end{tabular}
% \label{tab:label}
% \end{table}

% Reference to Table~\vref{tab:label}. % The \vref command specifies the location of the reference

% %------------------------------------------------

% \subsection{Figure Composed of Subfigures}

% Reference the figure composed of multiple subfigures as Figure~\vref{fig:esempio}. Reference one of the subfigures as Figure~\vref{fig:ipsum}. % The \vref command specifies the location of the reference

% \lipsum[15-18] % Dummy text

% \begin{figure}[tb]
% \centering
% \subfloat[A city market.]{\includegraphics[width=.45\columnwidth]{Lorem}} \quad
% \subfloat[Forest landscape.]{\includegraphics[width=.45\columnwidth]{Ipsum}\label{fig:ipsum}} \\
% \subfloat[Mountain landscape.]{\includegraphics[width=.45\columnwidth]{Dolor}} \quad
% \subfloat[A tile decoration.]{\includegraphics[width=.45\columnwidth]{Sit}}
% \caption[A number of pictures.]{A number of pictures with no common theme.} % The text in the square bracket is the caption for the list of figures while the text in the curly brackets is the figure caption
% \label{fig:esempio}
% \end{figure}

% %----------------------------------------------------------------------------------------
% %  BIBLIOGRAPHY
% %----------------------------------------------------------------------------------------

% \renewcommand{\refname}{\spacedlowsmallcaps{References}} % For modifying the bibliography heading

% \bibliographystyle{unsrt}

% \bibliography{sample.bib} % The file containing the bibliography

% %----------------------------------------------------------------------------------------

\end{document}
